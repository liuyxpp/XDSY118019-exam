\documentclass[UTF8]{ctexart}
\usepackage{amssymb} 
\usepackage{amsmath}
\usepackage{mathrsfs}
\usepackage{graphicx}

\author{任奕凝 21300180116}
\title{跨入科学研究之门 期末考试一 使用文档 }
\date{\today}
\begin{document}
	\maketitle
	
\section{问题描述及解答思路}
	\subsection{问题描述}
	编写python代码以实现Bisection算法和Newton算法,并将该两种算法与Numpy/Scipy中的求根函数进行比较。同时比较这两种算法在求解$x^2-2=0$在区间$(1,2)$中的根的收敛速度
	\subsection{解答思路}
	\begin{itemize}
		\item
		写出Bisection算法函数和Newton算法函数
		\item
		定义函数$f_{1}=2x-tan(x)$,$e^{x+1}-2-x$,$x^{-2}-sin(x)$,并用Scipy中的求根函数求根,再用以上定义的两种算法求根
		\item 
		用Bisection算法和Newton算法分别求解$x^2-2=0$在区间$(1,2)$中的根,记录每次迭代得到的近似解,并用matplotlib.pyplot以迭代次数为横轴,以每次得到的近似解为纵轴,绘制图片。
	\end{itemize}
	
\section{如何使用代码}
	问题一问题二把函数,导数(问题二),求解区间的左右端点和目标误差输入,可以输出解。
	
	问题三问题四可以直接运行代码得到结果。
\end{document}