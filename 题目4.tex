\documentclass[UTF8]{ctexart}
\usepackage[left=2cm,right=2cm,top=2cm,bottom=2cm]{geometry}
\usepackage{graphicx}
\usepackage{float}
\usepackage{amsmath,amssymb,amsfonts}

%设定行距、配置数据包
\begin{document}
\section*{题目4}
	\raggedright
	\textbf{Q:} Find the solution of the following equation with respect to $\theta$:
	
	\[
	A \cos \theta + B \sin \theta + C = 0
	\]
	
	\textbf{A:} \\
	Let $x_1 = \cos \theta$ and $x_2 = \sin \theta$, then the solution is given by the intersection of the circle and the line:
	

	\begin{align*}
		x_1^2 + x_2^2 &= 1 \\
		Ax_1 + Bx_2 + C &= 0
	\end{align*}
	
	We reformulate the equations in a parametric form:
	
	\[
	|x|^2 = 1
	\]
	\[
	x(t) = \mathbf{a} + t \mathbf{b}
	\]
	
	where $x = (x_1, x_2)$, $\mathbf{a} = (0, -\frac{C}{B})$, $\mathbf{b} = (-\frac{C}{A}, \frac{C}{B})$, and $t$ is a parameter. The intersection points satisfy the following equation:
	
	\[
	|\mathbf{a} + t \mathbf{b}|^2 = 1
	\]
	
	which can be solved for $t$ to find the intersection points:
	
	\[
	t_{1,2} = \frac{-\mathbf{a} \cdot \mathbf{b} \pm \sqrt{(\mathbf{a} \cdot \mathbf{b})^2 - |\mathbf{b}|^2 (|\mathbf{a}|^2 - 1)}}{|\mathbf{b}|^2}
	\]
	
\end{document}
